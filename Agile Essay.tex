% Please do not change the document class
\documentclass[11pt]{scrartcl}

% Please do not change these packages
\usepackage[hidelinks]{hyperref}
\usepackage[none]{hyphenat}
\usepackage{setspace}
\doublespace

% You may add additional packages here
\usepackage{amsmath}

% Please include a clear, concise, and descriptive title
\title{Should Scrum be Taught in Secondary School to Improve the Lack of Collaborative Learning and Team Work that Employers are Looking For?}

\date{November 14, 2016}

% Please do not change the subtitle
\subtitle{COMP150 - Agile Development Practice}

% Please put your student number in the author field
\author{1605913}



\begin{document}

\maketitle

\abstract{This paper looks at whether SCRUM should be used in higher education. Looking into topics such as Active Learning, SCRUM in higher education, Teachers and Applying Concepts To The Games Industry. Most students leave higher education without proper collaborative skills and team work. This paper looks at whether SCRUM should be used to promote this in University.  }


\section{Introduction}

This paper will discuss "Should SCRUM be Taught in Secondary School to Improve the Lack of Collaborative Learning and Team Work that Employers are Looking For?", This is an important query to answer because a lot of employers are looking for graduates who have been engaging in group activities and collaboratively learning \cite{hansen2006benefits}. SCRUM is a good way to promote group work as the only way to initiate a SCRUM is to have a group of people to work with. This paper will look at SCRUM in higher education, Teachers and Applying Concepts To The Game Industry. 

\section{Active Learning}

In a study centered around active learning the author found that lecturing was the main way of teaching in universities since universities were founded over 900 years ago. In the study the author found out that students who took studies which only included traditional lecture techniques, were 1.5  times more likely to fail their degree than students whose courses included active learning\cite{freeman2014active}. The author thinks that this study can also be applied to the SCRUM technique because of the collaborative associations that it promotes, this study found that failure rates for courses with traditional lectures increase 55\% more than the results found for active learning. 

\section{SCRUM in Higher Education}

In a study "On the Use of Scrum in Project Driven Higher Education", the authors implemented the use of SCRUM into a project involving three teams. For the three teams the teaching lecturer became the SCRUM master and the role of the product owner was assumed by a psychiatrist who recorded the results \cite{persson2011use}. The author thinks that this practice of making the teacher the product owner, will help the student to sculpt the project easier because the teacher will be able to convey their exact specifications of the product. The reason that this is important is because in another study \cite{kropp2016teaching} agile was applied to a scrum Lego project that the authors conducted regularly with their students. With the teacher being the product owner or SCRUM master, so that the students can build the Lego city from the teacher’s specific product vision. Furthermore, from the figures in Table 2 ("Barriers and Difficulties") of the same study, the most difficult obstacles to work with were customer interaction, skills, and culture. Almost all of these difficulties are overcome by making the teacher the product owner or SCRUM master.

\section{Teachers}

One of the, if not the most important part of any educational environment is the attitude of the teacher and the way they act towards their students. In one study by Hattie \cite{hattie2008visible} he talks about an experiment where small groups of students had "significantly more positive" results then learning as an individual. The student’s results were incredibly enhanced when the students had group work experience or were given instructions. This study backs up that SCRUM will increase results significantly to individual learning, it shows that studying in groups also gives the students the necessary skills that employers are looking for once they graduate. The study also found that the students had to have work and the teaching provided to them adapted to their level of ability. The use of SCRUM could of been very useful in this exercise because of how much the students would need to check with the scrum master and product owner, to see whether their work was good enough or matching the description of the product owners vision. 

\section{Applying Concepts To The Games Industry}

In the book "The Multiplayer Classroom: Designing Coursework as a Game" the author applies the concepts of "XP" (Experience Points) and Raids (group collaboration to usually fight bosses) to a classroom environment. He did this by making the classroom into a real-life RPG style game (see table on page 28\cite{sheldon2011multiplayer}). This is a great method of education as work is rewarded immediately through points which correspond to a level which represents a grade. This technique is called Gamification where elements from games are applied to other activities such as education. Using a technique called the "Hat of Knowledge" the author of the study made students line up and then shouted out a question, the 'pullers' would then wait for a signal from the guild (Team), they would then grab the hat, get an answer from the team and then officially answer the question. This technique gets students used to working within a team, the reason for this is because the 'pullers' would have to rely on their team mates for assistance, if they jumped the gun and ran for the hat too early or without a signal from their team, they would wait in vain as their guild tried to give them an answer.

\section{Conclusion}

In conclusion, the use of SCRUM can be easily applied to a higher education environment and improve the results of students who use this method. The use of SCRUM will help students to gain the skills that employers are looking for such as team work and collaborative learning. 

\bibliographystyle{ieeetran}
\bibliography{references}

\end{document}